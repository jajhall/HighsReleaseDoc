\documentclass[a4paper,12pt,twoside]{article}
\usepackage{xspace}
\usepackage{url}
\usepackage{relsize}
\usepackage{calc}
\usepackage{dimensions}

\newcommand{\Rplus}{\protect\hspace{-.1em}\protect\raisebox{.35ex}{\smaller{\smaller\textbf{+}}}}
\newcommand{\CMake}{\texttt{CMake}\xspace}
\newcommand{\HiGHS}{\texttt{HiGHS}\xspace}
\newcommand{\OpenMP}{\texttt{OpenMP}\xspace}
\newcommand{\HiGHStextURL}{http:/\!/www.HiGHS.dev/}%{http:/\!/www.highs.dev/}

\newcommand{\cmake}{\CMake}
\newcommand{\highs}{\HiGHS}
\newcommand{\openmp}{\OpenMP}

\newcommand{\C}{\mbox{C}\xspace}
\newcommand{\Cpp}{\mbox{C\Rplus\Rplus}\xspace}


\def\code#1{\texttt{#1}}


\title{
\HiGHS pre-release document}
\date{\today}
\author{J. A. J. Hall}

\flushbottom
\renewcommand{\familydefault}{\sfdefault}

\begin{document}

%\pagenumbering{roman}
\maketitle
\tableofcontents

\section{Introduction}

Although \HiGHS has been referring to Version 1.0.0 for a couple of
years, only now do we feel happy to create a formal release of Version
1.0.0. This document is for people who have been using \HiGHS and will
be using Version 1.0.0.

When we created \HiGHS two years ago, we tried to make design
decisions that were good for both us and our users. We are still
generally happy with these decisions but, as \HiGHS has developed and
we have listened to our users, some have been revised. Most of the
effect has been internal, but some changes will affect users and are
documented here.

Full documentation is available in the Version 1.0.0 release document.

\section{Model definition}
\HiGHS began as an LP solver, but later acquired a MIP solver, and now
has a QP solver. Within \Cpp, the original \code{HighsLp} class was
stretched to define MIPs, but a QP can't be argued to be a case of LP!
Hence we have introduced a \code{HighsModel} class that can be used to
define an LP, MIP or QP. Within the \code{Highs} class,
\code{HighsModel} is passed to \HiGHS as an argument to
\code{passModel}.

In the \C interface, for reasons discussed below, the definition of
the \code{highs\_passLp} method to pass an LP has changed. There are
also \C methods \code{highs\_passMip} and \code{highs\_passQp} to pass
MIP and QP problems.

\subsection{Row-wise constraint matrix}
Since solvers generally access the constraint matrix column-wise, this
was the originally the only orientation permitted by \HiGHS for model
defintion. However, it is generally preferable for modelling
interfaces and users to specify the constraint matrix row-wise. Hence
the \code{HighsLp} class has a \code{MatrixOrientation} member to
specify whether the matrix is held row-wise or column-wise. In the \C
interface, a \code{rowwise} parameter has been introduced.

\subsection{Compatibility}
For back-compatibility, the \code{HighsLp} class has been retained,
and it is still possible to pass an LP or MIP to \HiGHS by giving
\code{passModel} a \code{HighsLp} as an argument. However, to ensure
future compatibility, users should pass a \code{HighsModel}.

\section{Model return}
For \Cpp users, the \code{Highs} class contains a method
\code{getModel} that returns a const reference to the internal
\code{HighsModel}. The method \code{getLp} that returned a const
reference to the internal \code{HighsLp} has been deleted from the
\code{Highs} class.

\end{document}
