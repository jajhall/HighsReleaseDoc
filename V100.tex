\documentclass[a4paper,12pt,twoside]{article}
\usepackage{xspace}
\usepackage{url}
\usepackage{relsize}
\usepackage{calc}
\usepackage{dimensions}
\usepackage{rotating}
\usepackage{colortbl}

\input Alias

\title{
\HiGHS Version 1.0.0}
\date{\today}
\author{J. A. J. Hall}

\flushbottom
\renewcommand{\familydefault}{\sfdefault}

\begin{document}

%\pagenumbering{roman}
\maketitle
\tableofcontents

\section{Introduction}
%
\HiGHS was developed from the dual simplex solver of
Huangfu~\cite{HuHa13} by adding the interior point method solver (for
LP problems) of Schork~\cite{ScGo20}, the MIP solver of Gottwald, and
the convex QP solver of Feldmeier. Its presolve for LP and MIP was
written by Galabova and Gottwald, and its primal simplex solver was
written by Hall, based on work by Huangfu. Other features have been
written by Hall.

\HiGHS is mainly written in \Cpp with \OpenMP directives, but also has
some \C. The primary interface to \HiGHS is via \Cpp, but it also has full
interfaces for \C, \Csharp, \Fortran and \Python. There are also
third-party interfaces to \HiGHS from \Julia, \Rust and \Javascript.
This document describes the \Cpp interface and corresponding \C
interface methods. The other interfaces are analogous to the \C
interface.

\HiGHS uses \CMake as a build system and requires version 3.15. It has
been developed and tested on various Linux, MacOS and Windows
installations using both the GNU (\gpp) and Intel (\icc) \Cpp
compilers. Note that HiGHS requires (at least) version 4.9 of the GNU
compiler. If \OpenMP is unavailable to \HiGHS, then it will compile,
but will run in serial. \HiGHS has no third-party dependencies.

Although \HiGHS is freely available under the MIT license, we would be
pleased to learn about users' experience and give advice via email
sent to \HiGHSemail. If you use \HiGHS in an academic context, please
acknowledge this and cite the article of Huangfu and
Hall~\cite{HuHa13}, unless you make specific use of the interior point
method solver, in which case you should cite the work of Schork and
Gondzio~\cite{ScGo20}.

\subsection{Long integers}
%
\input HighsInt

\section{Problem definition}
%
Problems in \HiGHS are defined as an instance of the \code{HighsModel}
class. This consists of one instance of the \code{HighsHessian} class,
and one instance of the \code{HighsLp} class. Communication of models
to and from \HiGHS is possible via instances of the \code{HighsLp} or
\code{HighsModel} class. In the \C and other interfaces, communication
of models is via scalar values and addresses of vectors.
\subsection{The \code{HighsLp} class}
The \code{HighsLp} class allows LP and MIP problems of the form
$$
\begin{array}{rl}
  \mathrm{minimize/maximize}&f+\bfc^T\bfx\\
  \mathrm{subject~to}&\bfL\le A\bfx\le \bfU\\
    &\bfl\le\bfx\le\bfu\\
  &\{x_i: i\in\SetI\}\in\Z
  \end{array}
$$
to be defined, and has the following members.
\begin{itemize}
\item \code{\HighsInt \numCol}: Number of columns (variables) $\bfx$ in the model
\item \code{\HighsInt \numRow}: Number of rows (constraints) in the model
\item \code{std::vector<\HighsInt> \Astart}: Start of each constraint matrix vector in the compressed sparse storage
\item \code{std::vector<\HighsInt> \Aindex}: Indices of each constraint matrix vector in the compressed sparse storage
\item \code{std::vector<double> \Avalue}: Values of each constraint matrix vector in the compressed sparse storage
\item \code{std::vector<double> \colCost}: Cost (gradient) vector $\bfc$ in the objective function
\item \code{std::vector<double> \colLower}: Lower bounds $\bfl$ on the columns
\item \code{std::vector<double> \colUpper}: Upper bounds $\bfu$ on the columns
\item \code{std::vector<double> \rowLower}: Lower bounds $\bfL$ on the rows
\item \code{std::vector<double> \rowUpper}: Upper bounds $\bfU$ on the rows
\item \code{MatrixOrientation \orientation}: Orientation of the constraint matrix
\item \code{ObjSense \sense}: Sense of the objective: minimize or maximize
\item \code{double \offset}: Constant term $f$ in the objective function
\item \code{std::string \modelName}: The name of the model
\item \code{std::vector<std::string> \colName}: The names of the columns
\item \code{std::vector<std::string> \rowName}: The names of the rows
\item \code{std::vector<HighsVarType> \integrality}: The set $\SetI$ of integrality restrictions on columns
\end{itemize}


\subsubsection{Constraint matrix}
%
The constraint matrix is held using compressed sparse storage. Users
unfamiliar with this format should consult
Wikipedia~\cite{WikiSparseMatrix}. Since solvers generally access the
constraint matrix column-wise, this was originally the only
orientation permitted by \HiGHS. However, it is often preferable for
modelling interfaces and users to specify the constraint matrix
row-wise. The \orientation member of the \code{HighsLp}
class is used to specify whether the matrix is held row-wise or
column-wise. If the orientation is not specified, it is assumed to be
\inred{???-wise}. In the \C interface, an \HighsInt parameter indicates
whether the matrix is \inred{row-wise (2) or column-wise (1)}.

\subsubsection{Objective function}
%
The sense of the objective (minimize or maximize) is defined by the
\sense member of the \code{HighsLp} class. If the sense is not
specified, the objecitve is minimized. In the \C interface, an
\HighsInt parameter indicates whether the matrix is to be minimized
(1) or maximized (-1).

\subsubsection{Names}
%
When populating an instance of the \code{HighsLp} class, no name data
needs to be supplied, and name data cannot be supplied via the \C
interface. The \modelName member of the \code{HighsLp} class is the
name of the model, and is used occasionally in logging output. By
default it is an empty string. If the model is read from a data file,
then \modelName is the name of the file (less its extension). The
\colName and \rowName members of the \code{HighsLp} class are, by
default, of zero size. If the model is read from an MPS data file,
then \colName and \rowName contain its column and row names. If names
are available they are printed when \HiGHS writes the model or
solution.

\subsubsection{Integrality}
%
The \integrality member of the \code{HighsLp} class indicates which
variables (if any) must take integer values. If this member is of zero
size, or if all its entries are \code{HighsVarType::kContinuous}, the
set $\SetI$ is interpreted as being empty, so the instance is
interpreted as being an LP. If entries of \integrality are
\code{HighsVarType::kInteger} then, by default, the \HiGHS MIP solver
will be used to find the optimal integer values of the corresponding
variables. In the \C interface, an \HighsInt indicates whether a
particular variable is continous (0) or must take an integer value
(1).



\bibliographystyle{abbrv}
\bibliography{ref}


\end{document}
