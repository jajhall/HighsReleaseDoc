\documentclass[a4paper,12pt,twoside]{article}
\usepackage{xspace}
\usepackage{url}
\usepackage{relsize}
\usepackage{calc}
\usepackage{dimensions}
\usepackage{rotating}
\usepackage{colortbl}

\input Alias

\title{
\HiGHS pre-release document}
\date{\today}
\author{J. A. J. Hall}

\flushbottom
\renewcommand{\familydefault}{\sfdefault}

\begin{document}

%\pagenumbering{roman}
\maketitle
\tableofcontents

\section{Introduction}

Although \HiGHS has been referring to Version 1.0.0 for a couple of
years, only now do we feel happy to create a formal release of Version
1.0.0. When \HiGHS was created, we tried to make design decisions that
were good for both us and our users. We are still generally happy with
these decisions but, as \HiGHS has developed and we have listened to
our users, some have been revised. Most of the effect has been
internal, but some changes will affect users and are documented here.

This document is for people who are familiar with using \HiGHS and
want to know what changes and new features there are in Version
1.0.0. New users, and those wanting a greater level of detail, should
consult the full documentation in the Version 1.0.0 release document.

\section{Summary of enhancements in Version 1.0.0}
\HiGHS began as an LP solver, but later acquired a MIP solver, and now
has a QP solver. In addition to these major developments, here is a
list of new features, most of which have been developed at the request
of users. Full details are available in the Version 1.0.0 release
document.
\begin{itemize}
\item \HiGHS can be compiled to use long (64-bit) integers.
\item \HiGHS can receive or build the constraint matrix row-wise.
\item The dual values for rows are no longer negated.
\item Control of logging output from \HiGHS has been simplified.
\item When the model is an LP and has been solved to optimality,
  sensitivity ranges for all costs, active bounds and basic variables
  are now available.
\item When the solving of a model is interrupted without obtaining an
  optimal solution, a primal and dual solution is now available (where
  possible).
\item The status of any primal or dual solution returned by \HiGHS is
  indicated by the value of the \code{HighsInfo} entries
  \code{primal\_solution\_status} and
  \code{dual\_solution\_status}. These are interpreted using
  the values in Table~\ref{tab:HighsSolutionStatus}.
  \begin{table}[ht]
    \centerline{
      \begin{tabular}{|l|l|p{8cm}|}\hline
        Value & \Cpp constant & Interpretation\\\hline
        0 & \code{kSolutionStatusNone} & There is no solution available \\
        1 & \code{kSolutionStatusInfeasible} & An infeasible solution is available\\
        2 & \code{kSolutionStatusFeasible} &An feasible solution is available\\\hline
    \end{tabular}}
    \caption{Interpreting solution status values}\label{tab:HighsSolutionStatus}
  \end{table}
  
\item \HiGHS can now be used to presolve an LP or MIP, allowing users
  to extract the model. In a future release, \HiGHS will determine the
  optimal primal solution, dual solution or basis for the original
  model, given the corresponding data for the optimal solution of the
  presolved model.
\end{itemize}

\section{Long integers}
\input HighsInt

\section{Return status}

Since the names of constants in \HiGHS now conform to the Google \Cpp
style guide~\cite{GoogleStyleGuide}, numerous constants that have
changed name. Those used in the \Cpp interface are listed in
Section~\ref{sect:ChangedConstants}. However, the return status
\EnumClass \HighsStatus has changed both the names of its values and
their cast to integer. The latter affects the \C interface, and was
done so that the integer values are consistent with \C
conventions. The new names and the corresponding integer value are
listed in Table~\ref{tab:HighsStatusEnumClass}.

\input TableHighsStatusEnumClass

In the \Cpp interface, the

This means that, in the \C
interface, the corresponding methods return the integer cast of the
\HighsStatus value: 0 implies success, 1 implies a warning, and 2
implies an error. 

\section{Model handling}
Now that \HiGHS can solve MIPs and QPs, as well as LPs, model handling
has been generalized. It is also now possible to pass or build the
constraint matrix row-wise.

\subsection{Model definition}
Within \Cpp, the original \code{HighsLp} class was stretched to define
MIPs, but a QP can't be argued to be a generalisation< of LP!  Hence
we have introduced a \code{HighsModel} class that can be used to
define an LP, MIP or QP. A \code{HighsLp} instance is a member of he
class \code{HighsModel} class, allowing all uses of \code{HighsLp} to
be maintained. Within the \code{Highs} class, \code{HighsModel} may be
passed to \HiGHS as an argument to \code{passModel}.

In the \C interface, for reasons discussed below, the definition of
the \code{Highs\_passLp} method to pass an LP has changed. There are
also \C methods \code{Highs\_passMip} and \code{Highs\_passQp} to pass
MIP and QP problems.

\subsubsection{Row-wise constraint matrix}
Since solvers generally access the constraint matrix column-wise, this
was originally the only orientation permitted by \HiGHS. However, it
is often preferable for modelling interfaces and users to specify the
constraint matrix row-wise. Hence the \code{HighsLp} class has an
\orientation member to specify whether the matrix is held
row-wise or column-wise. In the \C interface, \code{Highs\_passLp} now
has a parameter to indicate whether the matrix is passed row-wise or
column-wise.

\subsubsection{Compatibility}
For back-compatibility, the \code{HighsLp} class has been retained,
and it is still possible to pass an LP or MIP to \HiGHS by giving
\code{passModel} a \code{HighsLp} instance as an argument. However, to
ensure future compatibility, users should pass a \code{HighsModel}.

\subsection{Model return}
For \Cpp users, the \code{Highs} class contains a method
\code{getModel} that returns a const reference to the internal
\code{HighsModel} instance. The method \code{getLp} that returns a
const reference to the internal \code{HighsLp} instance has been
retained.

\section{Model status}

The model status is a value that indicates the model's feasibility,
unboundedness or whether an optimal solution has been obtained by the
solver, as well as various (unlikely!) failure scenarios. The possible
statuses that can occur have changed for some edge cases. In particular, and consistent with
commercial solvers, if a model is both primal and dual infeasible,
\HiGHS no longer identifies this: it will only indicate primal
infeasibility. Again, as with commercial solvers, if a model is dual
infeasible then, unless it is identified as being either primal
infeasible, or primal unbounded, the \HiGHS model status will indicate
it as being ``infeasible or unbounded''. 

Within \Cpp, the model status value is taken from an \code{enum class}
\HighsModelStatus. Since the names of constants in \HiGHS now
conform to the Google \Cpp style guide~\cite{GoogleStyleGuide}, the entries of
\HighsModelStatus have changed. For the \C interface, entries of
\HighsModelStatus are cast to integers and, in general, the
value associated with a given status has changed. The correspondence
between current and former model status values is set out in
Table~\ref{tab:HighsModelStatus}.

\begin{sidewaystable}
\centerline{
  \begin{tabular}{|p{8cm}|r|l|r|l|}\hline
&\multicolumn2{c|}{Currently}&\multicolumn2{c|}{Formerly}\\\cline{2-5}
    Description&Integer&\HighsModelStatus&Integer&\HighsModelStatus\\\hline
Not set&0&\code{kNotset}&0&\code{NOTSET}\\
Error loading the model&1&\code{kLoadError}&1&\code{LOAD\_ERROR}\\
Error in the model definition&2&\code{kModelError}&2&\code{MODEL\_ERROR}\\
Error presolving the model&3&\code{kPresolveError}&3&\code{PRESOLVE\_ERROR}\\
Error solving the model&4&\code{kSolveError}&4&\code{SOLVE\_ERROR}\\
Error postsolving the model&5&\code{kPostsolveError}&5&\code{POSTSOLVE\_ERROR}\\
Model is empty&6&\code{kModelEmpty}&6&\code{MODEL\_EMPTY}\\
Model solution is optimal&7&\code{kOptimal}&9&\code{OPTIMAL}\\
Model is (primal) infeasible &8&\code{kInfeasible}&7&\code{PRIMAL\_INFEASIBLE}\\
Model is (primal) unbounded or infeasible&9&\code{kUnboundedOrInfeasible}&&\code{}\\
Model is (primal) unbounded&10&\code{kUnbounded}&8&\code{PRIMAL\_UNBOUNDED}\\
Any optimal objective is at least a given bound &11&\code{kObjectiveBound}&10&\code{REACHED\_DUAL\_OBJECTIVE\_VALUE\_UPPER\_BOUND}\\
There is a feasible solution with at least a given target value &12&\code{kObjectiveTarget}&&\code{}\\
The solution time limit has been reached&13&\code{kTimeLimit}&11&\code{REACHED\_TIME\_LIMIT}\\
The solution iteration limit has been reached&14&\code{kIterationLimit}&12&\code{REACHED\_ITERATION\_LIMIT}\\
The model status is unknown&15&\code{kUnknown}&&\code{}\\
Model is primal and dual infeasible&&&13&\code{PRIMAL\_DUAL\_INFEASIBLE}\\
Model is dual infeasible&&&14&\code{DUAL\_INFEASIBLE}\\\hline
  \end{tabular}}
\caption{Interpreting current and former model status values}\label{tab:HighsModelStatus}
\end{sidewaystable}

\section{Row dual values}
For historical reasons, the sign of the dual values for constraints
that were returned by \HiGHS were negated. This is not done in Version
1.0.0. For convenience and communication, the constant
$$
\code{kHighsPrereleaseRowDualSign}
$$
%
has been introduced into recent pre-release versions of \HiGHS. If it
is not present in the version you are using, or takes a value of 1,
then the row dual signs are negated. If it takes a value of -1, then
row dual signs are not negated. Thus, multiplying any \HiGHS row duals
by this value yields the negated row duals with which many users will
be familiar.

\section{Logging}
The original mechanism for handling logging output from \HiGHS was
over-engineered and opaque. It has now been re-written. By default,
log output goes to both the console and to the file
\file{Highs.log}. The production of log output and name of the logging
file is controlled by the \code{HighsOptions} settings in
Table~\ref{tab:HighsLogging}.
\begin{table}[ht]
\centerline{
  \begin{tabular}{|l|l|l|p{10cm}|}\hline
    Option name & Type & Default & Description\\\hline
    \code{output\_flag} & \code{bool} & \code{true} & Determines whether there is logging output from \HiGHS\\
    \code{log\_to\_console} & \code{bool} & \code{true} & Determines whether any logging output from \HiGHS goes to the console\\
    \code{log\_file} & \code{string} & \file{Highs.log} & Allows the name of the logging file to be changed
    %    \\\code{log\_dev\_level} & \HighsInt &
    \\\hline
    \end{tabular}}
\caption{Logging options}\label{tab:HighsLogging}
\end{table}


\section{Deleted options}

When \HiGHS is called from an application, the \code{model\_file} and
\code{options\_file} options are ambiguous and redundant. Hence they
have been deleted from \code{HighsOptions}. When \HiGHS is run as an
executable, these options are not redundant, so the corresponding
command-line options are retained.

The options \code{solution\_file}, \code{write\_solution\_to\_file}
and \code{write\_solution\_pretty} have been retained in
\code{HighsOptions}, but are only relevant when \HiGHS is run as an
executable, and are set in the options file. In an application, they
are redundant since their function is duplicated by parameters to
methods that write the solution to a file.

\section{Changed constants}~\label{sect:ChangedConstants}
Since the names of constants in \HiGHS now conform to the Google \Cpp
style guide~\cite{GoogleStyleGuide}, numerous constants that may be
used in \Cpp applications have changed name. With the exception of
\HighsStatus (see above), their values (when cast to integer) have not
changed. The current and former names, together with integer values
and short description are given in Table~\ref{tab:ChangedConstants}.

\begin{sidewaystable}
\centerline{
  \begin{tabular}{|l|l|r|p{10cm}|}\hline
    Currently & Formerly & Value & Description \\\hline
    \ObjSenseMinimize & \code{ObjSense::MINIMIZE} & -1 & Objective sense is minimize\\
    \ObjSenseMaximize & \code{ObjSense::MAXIMIZE} &  1 & Objective sense is maximize\\\hline
    \HighsBasisStatusLower & \code{HighsBasisStatus::LOWER} & 0 & Variable is at its lower bound (including fixed variables)\\
    \HighsBasisStatusBasic & \code{HighsBasisStatus::BASIC} & 1 & Variable is basic \\
    \HighsBasisStatusUpper & \code{HighsBasisStatus::UPPER} & 2 & Variable is at its upper bound\\
    \HighsBasisStatusZero & \code{HighsBasisStatus::ZERO} & 3 & Free variable is nonbasic and set to zero\\
    \HighsBasisStatusNonbasic & \code{HighsBasisStatus::NONBASIC} & 4 & Variable is nonbasic with no specific bound information\\\hline
    \end{tabular}}
\caption{Changed Constants}\label{tab:ChangedConstants}
\end{sidewaystable}

\section{Changed, deleted and deprecated methods}

To allow progress and keep the \HiGHS software engineering demands
within the resources available, some methods have been changed, a
small number have been deleted, and a rather larger number are
deprecated. These can be expected to be deleted in future versions of
\HiGHS.

\subsection{Changed methods}

For historical reasons, the methods in the \Cpp interface listed in
Table~\ref{tab:ChangedMethods} returned \Bool rather than
\HighsStatus, with \true indicating success. When cast to integer in
the \C interface, 0 implied an error, and 1 success: the opposite of
the \C convention! For consistency and clarity, they now all return
\HighsStatus.

\begin{table}[ht]
\centerline{
  \begin{tabular}{|l|p{8cm}|}\hline
\Cpp \HiGHS class & \C interface equivalent(s)\\\hline
\HighsAddRow & \CHighsAddRow\\\hline
\HighsAddRows & \CHighsAddRows\\\hline
\HighsAddCol & \CHighsAddCol\\\hline
\HighsAddCols & \CHighsAddCols\\\hline
\HighsChangeObjectiveSense & \CHighsChangeObjectiveSense\\\hline
\HighsChangeColIntegrality & \CHighsChangeColIntegrality
\\\hline
\HighsChangeColsIntegrality & 
\CHighsChangeColsIntegralityByRange
\CHighsChangeColsIntegralityBySet
\CHighsChangeColsIntegralityByMask
\\\hline
\HighsChangeColCost & \CHighsChangeColCost\\\hline
\HighsChangeColsCost & 
\CHighsChangeColsCostByRange
\CHighsChangeColsCostBySet
\CHighsChangeColsCostByMask
\\\hline
\HighsChangeColBounds & \CHighsChangeColBounds\\\hline
\HighsChangeColsBounds & 
\CHighsChangeColsBoundsByRange
\CHighsChangeColsBoundsBySet
\CHighsChangeColsBoundsByMask
\\\hline
\HighsChangeRowBounds & \CHighsChangeRowBounds\\\hline
\HighsChangeRowsBounds & 
\CHighsChangeRowsBoundsByRange
\CHighsChangeRowsBoundsBySet
\CHighsChangeRowsBoundsByMask
\\\hline
\HighsChangeCoeff & \CHighsChangeCoeff\\\hline
\HighsGetObjectiveSense & \CHighsGetObjectiveSense\\\hline
\HighsGetCols & 
\CHighsGetColsByRange
\CHighsGetColsBySet
\CHighsGetColsByMask
\\\hline
\HighsGetRows & 
\CHighsGetRowsByRange
\CHighsGetRowsBySet
\CHighsGetRowsByMask
\\\hline
\HighsGetCoeff & \CHighsGetCoeff\\\hline
\HighsDeleteCols & 
\CHighsDeleteColsByRange
\CHighsDeleteColsBySet
\CHighsDeleteColsByMask
\\\hline
\HighsDeleteRows & 
\CHighsDeleteRowsByRange
\CHighsDeleteRowsBySet
\CHighsDeleteRowsByMask
\\\hline
\HighsScaleCol & \CHighsScaleCol\\\hline
\HighsScaleRow & \CHighsScaleRow\\\hline
    \end{tabular}}
\caption{Methods now returning \HighsStatus in \Cpp, and cast of \HighsStatus in \C}\label{tab:ChangedMethods}
\end{table}

\subsection{Deleted methods}

Three methods have been deleted from the \C interface. These are
\code{Highs\_modelStatusToChar}, \code{Highs\_solutionStatusToChar}
and \code{Highs\_highsModelStatusToChar} since they fail on Windows
and MacOS (but not on Linux).

\subsection{Deprecated methods}

The deprecated methods in the \Cpp \HiGHS class, the corresponding
method(s) in the \C interface and their replacements (where relevant)
are listed in Table~\ref{tab:DeprecatedMethodsCpp}. The first two relate
to the old logging system, and both now set the option
\code{output\_flag} to false, stopping any logging from \HiGHS: they
will soon be deleted. Early versions of \HiGHS had a dedicated method
to get the simplex iteration count, and this was also returned by two
methods in the \C interface. However, this will be deleted now that
the simplex iteration count is one of many solution attributes are
available from \HiGHS via the \code{HighsInfo} class and, in the \C
interface, via \code{Highs\_getIntInfoValue} and
\code{Highs\_getDoubleInfoValue}. The remaining deprecated methods
have replacement methods with ``Highs'' omitted from the name. In the
case of the \HiGHS class, ``Highs'' is implied by the class name, and
in the \C interface, the method names are distinguished as belonging
to \HiGHS by beginning ``\code{Highs\_}''.

Deprecated methods in the \C interface are listed in Table~\ref{tab:DeprecatedMethodsC}

\begin{sidewaystable}
\centerline{
  \begin{tabular}{|l|p{7cm}|l|p{6cm}|}\hline
\multicolumn2{|c|}{Deprecated}&\multicolumn2{c|}{Replacement}\\\hline
\Cpp \HiGHS class & \C interface equivalent(s) & \Cpp & \C equivalent(s) \\\hline
\code{setHighsLogfile} & \code{Highs\_setHighsLogfile} & \multicolumn2{c|}{None: redundant due to logging change}\\
\code{setHighsOutput} & \code{Highs\_setHighsOutput} & \multicolumn2{c|}{None: redundant due to logging change}\\\hline
\code{getSimplexIterationCount} &
\code{Highs\_getSimplexIterationCount};
\code{Highs\_getIterationCount} & \code{getInfo} & \code{Highs\_getIntInfoValue} \\\hline
\code{setHighsOptionValue} &
\code{Highs\_setHighsBoolOptionValue};
\code{Highs\_setHighsIntOptionValue};
\code{Highs\_setHighsDoubleOptionValue};
\code{Highs\_setHighsStringOptionValue}
& \code{setOptionValue} &
\code{Highs\_setBoolOptionValue};
\code{Highs\_setIntOptionValue};
\code{Highs\_setDoubleOptionValue};
\code{Highs\_setStringOptionValue} \\
\code{readHighsOptions} & \code{Highs\_readHighsOptions} & \code{readOptions} & \code{Highs\_readOptions}\\
\code{passHighsOptions} & \code{Highs\_passHighsOptions} & \code{passOptions} & \code{Highs\_passOptions}\\
\code{getHighsOptionValue} &
\code{Highs\_getHighsBoolOptionValue};
\code{Highs\_getHighsIntOptionValue};
\code{Highs\_getHighsDoubleOptionValue};
\code{Highs\_getHighsStringOptionValue}
& \code{getOptionValue} &
\code{Highs\_getBoolOptionValue};
\code{Highs\_getIntOptionValue};
\code{Highs\_getDoubleOptionValue};
\code{Highs\_getStringOptionValue}\\
\code{getHighsOptions} & \code{Highs\_getHighsOptions} & \code{getOptions} & \code{Highs\_getOptions}\\
\code{resetHighsOptions} & \code{Highs\_resetHighsOptions} & \code{resetOptions} & \code{Highs\_resetOptions}\\
\code{writeHighsOptions} & \code{Highs\_writeHighsOptions} & \code{writeOptions} & \code{Highs\_writeOptions}\\
\code{getHighsInfo} & \code{Highs\_getHighsInfo} & \code{getInfo} & \code{Highs\_getInfo}\\
\code{getHighsInfoValue} & \code{Highs\_getHighsIntInfoValue}; \code{Highs\_getHighsDoubleInfoValue} & \code{getInfoValue} & \code{Highs\_getIntInfoValue}; \code{Highs\_getDoubleInfoValue} \\
\code{writeHighsInfo} & \code{Highs\_writeHighsInfo} & \code{writeInfo} & \code{Highs\_writeInfo}\\
\code{getHighsInfinity} & \code{Highs\_getHighsInfinity} & \code{getInfinity} & \code{Highs\_getInfinity}\\
\code{getHighsRunTime} & \code{Highs\_getHighsRunTime} & \code{getRunTime} & \code{Highs\_getRunTime}\\\hline
    \end{tabular}}
\caption{Deprecated methods}\label{tab:DeprecatedMethodsCpp}
\end{sidewaystable}

\begin{table}[ht]
\centerline{
  \begin{tabular}{|p{5cm}|p{9cm}|}\hline
Deprecated&Replacement\\\hline
\code{Highs\_call} & \code{Highs\_lpCall}; \code{Highs\_mipCall} ; \code{Highs\_qpCall} \\
\code{Highs\_runQuiet} & None: redundant due to logging change\\
\code{Highs\_getHighsOptionType}& \code{Highs\_getOptionType}\\\hline
    \end{tabular}}
\caption{Deprecated \C methods}\label{tab:DeprecatedMethodsC}
\end{table}


\bibliographystyle{abbrv}
\bibliography{ref}


\end{document}
