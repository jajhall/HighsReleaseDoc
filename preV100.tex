\documentclass[a4paper,12pt,twoside]{article}
\usepackage{xspace}
\usepackage{url}
\usepackage{relsize}
\usepackage{calc}
\usepackage{dimensions}
\usepackage{rotating}

\newcommand{\Rplus}{\protect\hspace{-.1em}\protect\raisebox{.35ex}{\smaller{\smaller\textbf{+}}}}
\newcommand{\CMake}{\texttt{CMake}\xspace}
\newcommand{\HiGHS}{\texttt{HiGHS}\xspace}
\newcommand{\OpenMP}{\texttt{OpenMP}\xspace}
\newcommand{\HiGHStextURL}{http:/\!/www.HiGHS.dev/}%{http:/\!/www.highs.dev/}

\newcommand{\cmake}{\CMake}
\newcommand{\highs}{\HiGHS}
\newcommand{\openmp}{\OpenMP}

\newcommand{\C}{\mbox{C}\xspace}
\newcommand{\Cpp}{\mbox{C\Rplus\Rplus}\xspace}


\def\code#1{\texttt{#1}}
\def\file#1{\texttt{#1}}


\title{
\HiGHS pre-release document}
\date{\today}
\author{J. A. J. Hall}

\flushbottom
\renewcommand{\familydefault}{\sfdefault}

\begin{document}

%\pagenumbering{roman}
\maketitle
\tableofcontents

\section{Introduction}

Although \HiGHS has been referring to Version 1.0.0 for a couple of
years, only now do we feel happy to create a formal release of Version
1.0.0. When \HiGHS was created, we tried to make design decisions that
were good for both us and our users. We are still generally happy with
these decisions but, as \HiGHS has developed and we have listened to
our users, some have been revised. Most of the effect has been
internal, but some changes will affect users and are documented here.

This document is for people who are familiar with using \HiGHS and
want to know what changes and new features there are in Version
1.0.0. New users and those wanting a greater level of detail should
consult the full documentation in the Version 1.0.0 release document.

\section{Summary of enhancements in Version 1.0.0}
\HiGHS began as an LP solver, but later acquired a MIP solver, and now
has a QP solver. In addition to these major developments, here is a
list of new features, most of which have been developed at the request
of users. Full details are available in the Version 1.0.0 release
document.
\begin{itemize}
\item \HiGHS can be compiled to use long (64-bit) integers.
\item \HiGHS can receive or build the constraint matrix row-wise.
\item The dual values for rows are no longer negated.
\item Control of logging output from \HiGHS has been simplified.
\item When the model is an LP and has been solved to optimality,
  sensitivity ranges for all costs, active bounds and basic variables
  are now available.
\item When the solving of a model is interrupted without obtaining an
  optimal solution, a primal and dual solution now available (where
  possible).
\item The status of any primal or dual solution returned by \HiGHS is
  indicated by the value of the \code{HighsInfo} entries
  \code{primal\_solution\_status} and
  \code{dual\_solution\_status}. These are interpreted using
  the values in Table~\ref{tab:HighsSolutionStatus}.
  \begin{table}[h]
    \centerline{
      \begin{tabular}{|l|l|p{8cm}|}\hline
        Value & \Cpp constant & Interpretation\\\hline
        0 & \code{kSolutionStatusNone} & There is no solution available \\
        1 & \code{kSolutionStatusInfeasible} & An infeasible solutiuon is available\\
        2 & \code{kSolutionStatusFeasible} &An feasible solutiuon is available\\\hline
    \end{tabular}}
    \caption{Interpreting solution status values}\label{tab:HighsSolutionStatus}
  \end{table}
  
\item \HiGHS can now be used to presolve an LP or MIP, allowing users
  to extract the model. In a future release, \HiGHS will determine the
  optimal primal solution, dual solution or basis for the original
  model given the corresponding data for the optimal solution of the
  presolved model.
\end{itemize}

\section{Long integers}
To allow larger models to be handled, it is now possible to compile
\HiGHS so that it uses long (64-bit) integers in all cases. When built
using \code{CMake}, this is achieved by setting
\code{-DHIGHSINT64=on}. The type \code{HighsInt} is then set to be
\code{int64\_t} rather than \code{int}. By default, \HiGHS compiles
with standard (32-bit) integers. Long integers are not currently
supported by the \code{FORTRAN} interface. Since \HiGHS can be
compiled with either standard or long integers, integer values are
referred to below anonymously as \code{integer}.

\section{Model handling}
Now that \HiGHS can solve MIPs and QPs, as well as LPs, model handling
has been generalised. It is also now possible to pass or build the
constraint matrix row-wise.

\subsection{Model definition}
Within \Cpp, the original \code{HighsLp} class was
stretched to define MIPs, but a QP can't be argued to be a case of LP!
Hence we have introduced a \code{HighsModel} class that can be used to
define an LP, MIP or QP. Within the \code{Highs} class,
\code{HighsModel} is passed to \HiGHS as an argument to
\code{passModel}.

In the \C interface, for reasons discussed below, the definition of
the \code{highs\_passLp} method to pass an LP has changed. There are
also \C methods \code{highs\_passMip} and \code{highs\_passQp} to pass
MIP and QP problems.

\subsubsection{Row-wise constraint matrix}
Since solvers generally access the constraint matrix column-wise, this
was originally the only orientation permitted by \HiGHS. However, it
is often preferable for modelling interfaces and users to specify the
constraint matrix row-wise. Hence the \code{HighsLp} and
\code{HighsModel} classes have a \code{MatrixOrientation} member to
specify whether the matrix is held row-wise or column-wise. In the \C
interface, a \code{rowwise} parameter has been introduced to the
parameter list of \code{highs\_passLp}.

\subsubsection{Compatibility}
For back-compatibility, the \code{HighsLp} class has been retained,
and it is still possible to pass an LP or MIP to \HiGHS by giving
\code{passModel} a \code{HighsLp} as an argument. However, to ensure
future compatibility, users should pass a \code{HighsModel}.

\subsection{Model return}
For \Cpp users, the \code{Highs} class contains a method
\code{getModel} that returns a const reference to the internal
\code{HighsModel}. The method \code{getLp} that returned a const
reference to the internal \code{HighsLp} has been deleted from the
\code{Highs} class.

\section{Model status}

The model status is a value that indicates the model's feasibility,
unboundedness or whether an optimal solution has been obtained by the
solver. The possible statuses that can occur, and the values that
indicate the status, have changed for some edge cases. In particular,
and consistent with commercial solvers, if a model is both primal and
dual infeasible, \HiGHS no longer idetifies this: it will only
indicate primal infeasibility. Again, as with commercial solvers, if a
model is dual infeasible then, unless it is identified as being either
primal infeasible, or primal unbounded, the \HiGHS model status will
indicate it as being infeasible or unbounded.

Within \Cpp the model status value is taken from an enum class
\code{HighsModelStatus}. Since the names of constants in \HiGHS now
conform to the Google \Cpp style guide~\cite{GoogleStyleGuide}, the entries of
\code{HighsModelStatus} have changed. For the \C interface, entries of
\code{HighsModelStatus} are cast to integers and, in general, the
value associated with a given status has changed. The correspondence
between current and former model status values is set out in
Table~\ref{tab:HighsModelStatus}.

\begin{sidewaystable}
\centerline{
  \begin{tabular}{|p{8cm}|l|l|l|l|}\hline
&\multicolumn2{c|}{Currently}&\multicolumn2{c|}{Formerly}\\\cline{2-5}
    Description&Integer&\code{HighsModelStatus}&Integer&\code{HighsModelStatus}\\\hline
Not set&0&\code{kNotset}&0&\code{NOTSET}\\
Error loading the model&1&\code{kLoadError}&1&\code{LOAD\_ERROR}\\
Error in the model definition&2&\code{kModelError}&2&\code{MODEL\_ERROR}\\
Error presolving the model&3&\code{kPresolveError}&3&\code{PRESOLVE\_ERROR}\\
Error solving the model&4&\code{kSolveError}&4&\code{SOLVE\_ERROR}\\
Error postsolving the model&5&\code{kPostsolveError}&5&\code{POSTSOLVE\_ERROR}\\
Model is empty&6&\code{kModelEmpty}&6&\code{MODEL\_EMPTY}\\
Model solution is optimal&7&\code{kOptimal}&9&\code{OPTIMAL}\\
Model is (primal) infeasible &8&\code{kInfeasible}&7&\code{PRIMAL\_INFEASIBLE}\\
Model is (primal) unbounded or infeasible&9&\code{kUnboundedOrInfeasible}&&\code{}\\
Model is (primal) unbounded&10&\code{kUnbounded}&8&\code{PRIMAL\_UNBOUNDED}\\
Any optimal objective is at least a given bound &11&\code{kObjectiveBound}&10&\code{REACHED\_DUAL\_OBJECTIVE\_VALUE\_UPPER\_BOUND}\\
There is a feasible solution with at least a given target value &12&\code{kObjectiveTarget}&&\code{}\\
The solution time limit has been reached&13&\code{kTimeLimit}&11&\code{REACHED\_TIME\_LIMIT}\\
The solution iteration limit has been reached&14&\code{kIterationLimit}&12&\code{REACHED\_ITERATION\_LIMIT}\\
The model status is unknown&15&\code{kUnknown}&&\code{}\\
Model is primal and dual infeasible&&&13&\code{PRIMAL\_DUAL\_INFEASIBLE}\\
Model is dual infeasible&&&14&\code{DUAL\_INFEASIBLE}\\\hline
  \end{tabular}}
\caption{Interpreting current and former model status values}\label{tab:HighsModelStatus}
\end{sidewaystable}

\section{Row dual values}
For historical reasons, the sign of the dual values for constraints
that were returned by \HiGHS were negated. This is not done in Version
1.0.0. For convenience and communication, the constant
$$
\code{kHighsPrereleaseRowDualSign}
$$
%
has been introduced into recent pre-release versions of \HiGHS. If it
is not present in the version you are using, or takes a value of 1,
then the row dual signs are negated. If it takes a value of -1, then
row dual signs are not negated. Thus, multiplying any \HiGHS row duals
by this value yields the negated row duals with which many users will
be familiar.

\section{Logging}
The original mechanism for handling logging output from \HiGHS was
over-engineered and opaque. It has now been re-written. By default,
log output goes to both the console and to the file
\file{Highs.log}. The production of log output and name of the logging
file is controlled by the \code{HighsOptions} settings in
Table~\ref{tab:HighsLogging}.
\begin{table}
\centerline{
  \begin{tabular}{|l|l|l|p{10cm}|}\hline
    Option name & Type & Default & Description\\\hline
    \code{output\_flag} & \code{bool} & \code{true} & Determines whether there is logging output from \HiGHS\\
    \code{log\_to\_console} & \code{bool} & \code{true} & Determines whether any logging output from \HiGHS goes to the console\\
    \code{log\_file} & \code{string} & \file{Highs.log} & Allows the name of the logging file to be changed
    %    \\\code{log\_dev\_level} & \code{integer} &
    \\\hline
    \end{tabular}}
\caption{Logging options}\label{tab:HighsLogging}
\end{table}


\section{Deleted options}

When \HiGHS is called from an application, the options
\code{model\_file} and \code{options\_file} are ambiguous and
redundant. Hence they have been deleted from \code{HighsOptions}. When
\HiGHS is run as an executable, these options are not redundant, so
the corresponding command-line options are retained.

The options \code{solution\_file}, \code{write\_solution\_to\_file}
and \code{write\_solution\_pretty} have been retained in
\code{HighsOptions}, but are only relevant when \HiGHS is run as an
executable, and are set in the options file. In an application, they
are redundant since their function is duplicated by parameters to
methods that write the solution to a file.

\section{Changed constants}
Since the names of constants in \HiGHS now conform to the Google \Cpp
style guide~\cite{GoogleStyleGuide}, numerous constants that may be
used in \Cpp applications have changed name. Their values (when cast
to integer if necessary) have not changed. The current and former
names, together with integer values and short desription are given in
Table~\ref{tab:ChangedConstants}.

\begin{sidewaystable}
\centerline{
  \begin{tabular}{|l|l|r|p{10cm}|}\hline
    Currently & Formerly & Value & Description \\\hline
    \code{ObjSense::kMinimize} & \code{ObjSense::MINIMIZE} & -1 & Objective sense is minimize\\
    \code{ObjSense::kMaximize} & \code{ObjSense::MAXIMIZE} &  1 & Objective sense is maximize\\\hline
     \code{HighsBasisStatus::kLower} & \code{HighsBasisStatus::LOWER} & 0 & Variable is at its lower bound (including fixed variables)\\
     \code{HighsBasisStatus::kBasic} & \code{HighsBasisStatus::BASIC} & 1 & Variable is basic \\
     \code{HighsBasisStatus::kUpper} & \code{HighsBasisStatus::UPPER} & 2 & Variable is at its upper bound\\
     \code{HighsBasisStatus::kZero} & \code{HighsBasisStatus::ZERO} & 3 & Free variable is nonbasic and set to zero\\
     \code{HighsBasisStatus::kNonbasic} & \code{HighsBasisStatus::NONBASIC} & 4 & Variable is nonbasic with no specific bound information\\\hline
    \end{tabular}}
\caption{Changed Constants}\label{tab:ChangedConstants}
\end{sidewaystable}

\section{Deleted and deprecated methods}



\bibliographystyle{abbrv}
\bibliography{ref}


\end{document}
